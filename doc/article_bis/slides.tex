 \documentclass[14pt]{beamer}
\usepackage[T1]{fontenc} 
\usepackage[utf8]{inputenc}
\usepackage[french]{babel}
\usepackage{color}

\setbeamertemplate{blocks}[rounded]

\usetheme{CambridgeUS}
\hypersetup{
  colorlinks,
  citecolor=black,
  filecolor=black,
  linkcolor=black,
  urlcolor=black
}

\title{Logoot}
\subtitle{a Scalable Optimistic Replication Algorithm for Collaborative Editing on P2P Networks}
\author{Adrien Drouet\\
        Alexandre Prenza}

\begin{document}
  %%%%%%%%%%%%%%%%%%%%%%%%%%%%%%%%%%%%%%%%%%%%%%%%%%%%%%%%%%%%%%%%%%%%%%%%%%%%%%
  %% Titre
  %%%%%%%%%%%%%%%%%%%%%%%%%%%%%%%%%%%%%%%%%%%%%%%%%%%%%%%%%%%%%%%%%%%%%%%%%%%%%%
  \begin{frame}
    \titlepage
  \end{frame}

	\begin{frame}
		\frametitle{Introduction}
		\begin{itemize}
			\item<1-> Collaborative editing
			\item <2->Wikipedia
				\begin{itemize}
					\item<3-> 10 million of articles
					\item<3-> centralized infrastructure
					\begin{itemize}
						\item<4-> fault tolerance
						\item<4-> costly scalability
					\end{itemize}
				\end{itemize}
			\item<5-> Solution ?
			\begin{itemize}
						\item<6-> Peer-to-peer
			\end{itemize}
		\end{itemize}
	\end{frame}

	\begin{frame}
		\frametitle{P2P}
		Peer-to-Peer constraint
		\begin{itemize}
			\item churn
			\item unknow/unbouded set of peer
		\end{itemize}
		
		Collaborative and real-time editor :
		\begin{itemize}
			\item<2-> Causality : all operations are ordered
			\item<2-> Convergence : when the system is idle, it's converges
			\item<2-> Intention : expected effect observed on all replica
			\item<2-> Scalability
		\end{itemize}
	\end{frame}

	\begin{frame}
	  \frametitle{Related Work}
		\begin{itemize}
			\item<1-> WOOKIE : Hasse diagramm extension
			\item<1-> TreeDoc : Binary tree
			\item<2-> use of tombstones
		\end{itemize}
	\end{frame}

	\begin{frame}
		\frametitle{Proposition}
		\begin{itemize}
			\item Logoot
			\begin{itemize}
				\item linear structure
				\item total order between elements
			\end{itemize}
			\item two operations
			\begin{itemize}
				\item insert(pid,text)
				\item delete(pid)
			\end{itemize}
		\end{itemize}
	\end{frame}

	\begin{frame}
		\frametitle{Logoot model - Identifier}
		Identifier
		\begin{itemize}
			\item couple (pid, content)
			\item $pid=pid_1.pid_2...pid_n.(pos, site)$
			\item total order :
				\begin{itemize}
					\item $(1,1) < (1,3)$
					\item $(1,1)(5,2) < (1,1)(14,1) < (4,2) $
				\end{itemize}
		\end{itemize}		
	\end{frame}

	\begin{frame}
		\frametitle{Logoot model - Modifying and Integrating}
		\begin{itemize}
			\item Modifying
				\begin{itemize}
					\item Generate one or more identifier
					\item postion can become bigger
				\end{itemize}
			\item Integration
				\begin{itemize}
					\item Binary search
					\item logarithmic time
					\item delete : no need of tombstone just delete from text
				\end{itemize}		
		\end{itemize}
	\end{frame}

	\begin{frame}
		\frametitle{Result}
		\begin{itemize}
			\item methodology
				\begin{itemize}
					\item Wooto, TreeDoc and Logoot
					\item Top edited and biggest page of wikipedia
					\item last 100 edits
				\end{itemize}
			\item Observation of overhead
		\end{itemize}
		
	\end{frame}

	\begin{frame}
		\frametitle{Result}
			Logoot observation :
			\begin{itemize}
					\item relative overhead is constant
					\item better than others when number of patch > 10.000 
			\end{itemize}
			Why ?
			\begin{itemize}
				\item No tombstone to maintain order
				\item A lot of delete operation (Update = delete + insert)
			\end{itemize}
	\end{frame}

	\begin{frame}
		\frametitle{Conclusion}
			\begin{itemize}
				\item behavior of real-time user is different than Wikipedia user
				\item but validate the logoot algorithme in practice
			\end{itemize}
	\end{frame}

\end{document}

