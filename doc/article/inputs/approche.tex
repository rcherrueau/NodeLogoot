 The goal of this experiments is to select some algorithms based on optimistic replications and evaluate them in a decentralized real-time collaborative editing system. The evaluations will be based on real context on the same conditions and using the same data flow.

The first approach is \emph{Operation Transformation}. This approach takes into account the effect of concurrent operations. An \emph{Operation Transformation} is used to keep document consistency after several concurrent operations. Google Docs use an algorithm named \emph{Jupiter} and a vector clock to detect concurrent operations, but this solution doesn’t get on well in the peer-to-peer system.

A new approach called \emph{Commutative Replicated Data Types} (CRDT) for peer-to-peer environment was introduced as a new class of replication mechanisms to preserve consistency. \emph{CRDT} doesn't require concurrent operation detections because it's designed for concurrent operations to be natively commutative.