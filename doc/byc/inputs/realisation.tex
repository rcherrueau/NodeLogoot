\section{R\'ealisation}
\begin{frame}{Réalisation}
\begin{description}
\item [Logoot] ~ 
  \begin{itemize}
    \item Algorithme d'édition collaborative de documents dans un réseau
    pair à pair
    \item Développé par l'équipe GDD (LINA) ces dernières années
    \item Utilisé par le noyau pour faire l'édition collaborative
  \end{itemize}
\item [Criojo] ~
  \begin{itemize}
    \item Langage de programmation chimique répartie
    \item En cours de développement par l'équipe ASCOLA (inria)
    \item Utilisé pour la distribution des modifications et la définitions
    des structures de documents
  \end{itemize}
\end{description}
\end{frame}

\begin{frame}{Logoot}
TODO: exemple
\end{frame}

\begin{frame}{Criojo}
Langage de programmation chimique répartie ?
\begin{itemize}
  \item Définis des agents répartis qui consomment et produisent des messages
  sur le réseau
    \begin{itemize}
    \item [$\Rightarrow$] Au niveau du noyau, chaque éditeur d'un même document
    est représenté par un agent chimique.
    \end{itemize}
  \item Représente l'état d'un agent comme une structure logique formée d'atomes
    \begin{itemize}
    \item [$\Rightarrow$] Au niveau du noyau, ce formalisme est particulièrement
    adapté à la représentation de documents
    \end{itemize}
\end{itemize}
\end{frame}

\begin{frame}{Criojo}
TODO: exemple
\end{frame}

\begin{frame}{Réalisation}
TODO: exemple
\end{frame}


