\section{Conception du projet}
\begin{frame}{Les deux technologies}
\begin{description}
\item [Logoot] : Algorithme d'édition collaborative de documents dans un réseau
    pair à pair
  \begin{itemize}
    \item Développé par l'équipe GDD (Inria, LINA) ces dernières années ;
    \item Utilisé par le noyau pour faire l'édition collaborative.
  \end{itemize} ~
\item [Criojo] : Langage de programmation \emph{chimique répartie}
  \begin{itemize}
    \item En cours de développement par l'équipe ASCOLA (Inria, LINA) ;
    \item Utilisé pour la distribution des modifications et la définition
    des documents.
  \end{itemize}
\end{description}
\end{frame}

\begin{frame}{Logoot}
\begin{itemize}
  \item Gère les conflits d'édition dans un document distribué ;
  \item Génère des identifiants pour le partage des modifications entre les
  collaborateurs d'un document ;
  \item Analyse les identifiants reçus pour positionner les modifications dans
  le document.
\end{itemize}
\end{frame}

\begin{frame}{Criojo}
Langage de programmation \emph{chimique répartie}
\begin{itemize}
  \item Définit des agents répartis qui consomment et produisent des messages
  sur le réseau ;
  \item Représente l'état d'un agent comme une structure logique formée
  d'atomes.
\end{itemize}
\end{frame}

\begin{frame}{Criojo}
TODO: exemple
\end{frame}

\begin{frame}{État des lieux}
TODO: état
\end{frame}

\begin{frame}{Ambition du projet}
\begin{itemize}
  \item Composer deux technologies provenant du monde de la Recherche
  \item \textbf{Pourquoi ?} Pour produire un programme ayant une utilité claire.
  Par exemple, 
  % Dire ici, par exemple à la fin du projet, faire du java dans eclipse de 
  % manière collaborative.
  \item \textbf{Comment ?} En utilisant les méthodes agiles (Scrum)
    \begin{itemize}
    \item Définir des cycles très courts de développement
    \item Fixer des réunions en fin de cycle avec un encadrant GDD, ASCOLA ou
    AtlanMod
    \end{itemize}
\end{itemize}
\end{frame}

