\section{Objectifs}

\subsection{Scénario}
\begin{frame}{Objectifs}
  \begin{block}{Scénario}
  On est samedi et je demande de l'aide à l'un de mes professeur pour corriger 
  le PowerPoint que je présenterais à la soutenance Boost Your Code.\\
  ~\\
  \emph{Comment je fais pour rendre le PowerPoint accessible à mon professeur et
  ainsi permettre la collaboration ?}
  \end{block}
\end{frame}

\subsection{Solutions}
\begin{frame}{Subversion}
  Subervsion :
  \begin{itemize}
    \item Logiciel de gestion de versions
    \item Stocke un ensemble de fichiers en conservant la chronologie des
    modifications
    \item Fait pour \textbf{le travail en équipe}
  \end{itemize}~

  Pour notre scénario :
  \begin{itemize}
    \item Le PowerPoint est sur un serveur (centralisé)
    \item Les modifications se font en deux temps (différé)
  \end{itemize}
\end{frame}

\begin{frame}{Git}
  Git :
  \begin{itemize}
    \item Logiciel de gestion de versions
    \item À l'inverse de subversion, un système \textbf{décentralisé} 
  \end{itemize}~

  Pour notre scénario :
  \begin{itemize}
    \item Chaque collaborateur à une copie du PowerPoint (décentralisé)
    \item Les modifications se font en deux temps (différé)
  \end{itemize}
\end{frame}

\begin{frame}{Google Docs}
  Google Docs :
  \begin{itemize}
    \item Plateforme pour le travail en ligne et collaboratif
    \item Permet l'édition de document Word, Excel et PowerPoint
  \end{itemize}~

  Pour notre scénario :
  \begin{itemize}
    \item Le PowerPoint est sur un serveur (centralisé)
    \item Les modifications se font en instantanées (temps réel)
  \end{itemize}
\end{frame}

\begin{frame}{Objectifs}
  \begin{description}
    \item[Svn] : Document centralisé, Collaboration différée 
    \item[Git] : Document décentralisé, Collaboration différée
    \item [Google Docs] : Document centralisé, Collaboration temps réel.
    \item<2-> [CEK-P2P] : Document décentralisé, Collaboration temps réel.
  \end{description}
\end{frame}

\begin{frame}[containsverbatim]{Objectifs}
\begin{verbatim}
  TODO: Justifier l'objectif NOYAU en parlant des formats de
  documents.
  
  Profiter de la justification du noyau pour dire :
    * Principal médium = édition
    * Second médium = profiter du réseau, exemple chat.
\end{verbatim}
\end{frame}

