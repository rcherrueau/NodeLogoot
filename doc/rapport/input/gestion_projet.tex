\chapter{Gestion de projet}
  Dans le cadre de ce projet, nous sommes cinq étudiants à nous répartir le 
  travail. Dans un premier temps, l'objectif était d'explorer un ensemble de 
  technologies (\emph{Dart}, \emph{GWT} et \emph{Javascript}).
  
  Après avoir implémenté le moteur logoot dans ces différents langages, nous
  nous sommes recentrés sur \emph{GWT}. Nous sommes donc trois à continuer de 
  travailler sur cette technologie. Une personne continue à experimenter 
  \emph{Dart} par curiosité pour ce nouveau langage. Afin d'intégrer une couche
  pair à pair au projet, la dernière personne se concentre sur \emph{Gossip} puis
  sur \emph{Jxta}.
  
  A la fin du projet, les limites induitent par \emph{GWT} nous ont amené à nous
  concentré sur une version \emph{HTML 5}. Cette version a alors deux ambitions :
  \begin{itemize}
	\item Montrer que l'HTML 5 permet les accès suffisant pour concevoir un 
	éditeur simplement.
	\item Intégrer le modèle \emph{logoot} directement dans le document.
  \end{itemize}
  
  Afin de gérer ce projet collaboratif, nous utilisons la plateforme 
  \emph{Github}\footnote{https://github.com/xZwop/R5A} en tant que 
  gestionnaire de versions, wiki et gestionnaire de tickets.
