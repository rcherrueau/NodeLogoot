\section{Implémentation en \emph{Java} \emph{GWT}}
  \subsection{Qu'est-ce que \emph{GWT}}
  Google Web Toolkit est un ensemble d'outils fournis par Google pour le
   développement d'application web complexe. C'est un \emph{framework} gratuit
   et open source.
   
   Lors du développement, le code est écrit en  \emph{Java}. Ensuite \emph{GWT} 
   inclus un compilateur qui le traduit en \emph{Javascript} afin de le rendre 
   compatible avec tous les navigateur du marché.
   
  \subsection{Objectif de l'implémentation en \emph{Java} \emph{GWT}}
  L'avantage d'utiliser \emph{GWT} est la possibilité de decouper notre 
  logiciel de facon complétement modulaire. En effet, nous respectons la
  conception du chapitre précédent \ref{chap:Conception}. Grâce à cela, 
  l'interface utilisateur, la gestion du réseau et le moteur \emph{logoot}
  sont totalement découplés les uns des autres. L'objectif de départ étant
  de fournir deux interfaces : une en client leger (web avec \emph{Javascript})
  et une en client lourd (avec \emph{Swing}\footnote{Swing est une bibliothèque
  graphique pour \emph{Java}}). 
  
  \subsection{État d'avancement}
	L'implentation est complétement fonctionnel mais nous nous sommes heurtés
	à de nombreux problèmes.
	
	\label{sec:textarea}
	Pour implémenter l'éditeur, on utilise un \emph{TextArea}. C'est le 
	composant de base en \emph{html} pour ce genre de tâches. Cet élément 
	est trés contraignant. En effet la gestion du curseur est complexe
	pour des raisons de sécurité (afin d'éviter une intrusion trop forte côté 
	client). Afin de changer le texte quand des changement sont effectué à 
	l'exterieur, nous somme obligé de redéfinir tout le texte après un 
	passage dans un algorithme de \emph{diff}\footnote{permet la comparaison 
	de texte en trouvant les différences}/\emph{patch}\footnote{integration
	d'un resultat de diff pour introduire les modifications}. Ce procédé
	est lent et a tendance à nous faire perdre certains caractères lorsque
	l'on tape rapidement.
	
	Le protocol clasique de communication utilisé sur le web n'est pas 
	adapté à notre projet. En général, ce sont les pages web qui requêtent
	le serveur et celui-ci réponds. Aucun problème pour envoyer nos modifications 
	au serveur via un appel RPC fournis par \emph{GWT}. Le problème se présente
	pour permettre la réception des modifications. Plusieurs solutions 
	techniques sont disponible :
	\begin{itemize}
		\item \emph{Server Push} : La méthode consiste à ne pas clore la
		transaction coté serveur. Aprés une requête du client, le serveur
		répond quand il le souhaite. Afin de permetre une connection continue 
		entre le client et le serveur, le client doit à la réception d'une 
		réponse émettre une nouvelle requête.
		\item \emph{Web Socket}\cite{websocket_spec} : Il s'agit d'une spécification en cours au sein
		du \emph{WHATWG}\footnote{Web Hypertext Application Technology Working 
		Group} afin de permettre une communication bidirectionnel entre client
		et serveur. En raison de failles de sécurité et du coup d'implémentation 
		timide dans les navigateurs, nous n'avons pas retenu cette solution.
		\item \emph{Server Sent Event} \cite{server_sent_spec} 
		\cite{server_sent_html5_rocks} 
		 : Cette technologie est également en
		cours de spécification. Elle permet au client d'initier une connexion
		au serveur et de lui permettre d'envoyer les informations quand il
		le souhaite sans clore la connection.
	\end{itemize}
	
	Nous avons choisi d'utiliser les Server Sent Event pour leurs simplicités.
	Par contre, \emph{GWT} ne permet d'utiliser directement cette fonction.
	On utilise alors un mécanisme qui permet d'inclure du \emph{Javascript} natif.
	Là encore, on touche à la limite de ce que permet \emph{GWT}.
