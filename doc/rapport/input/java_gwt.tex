\section{Implémentation en \emph{Java} \emph{GWT}}
  \subsection{Qu'est-ce que \emph{GWT}}
  Google Web Toolkit est un ensemble d'outils fournis par Google pour le
   développement d'application web complexe. C'est un outil gratuit et 
   open source.
   
   Lors du développement, le code est écrit en  \emph{Java}. Ensuite \emph{GWT} inclus un 
   compilateur qui le traduit en \emph{Javascript} afin de le rendre compatible
   avec tous les navigateur du marché.
   
  \subsection{Objectif de l'implémentation en \emph{Java} \emph{GWT}}
  L'avantage d'utilser \emph{GWT} est la possibilité de decouper notre 
  logiciel de facon complétement modulaire. En effet, nous respectons la
  conception du chapitre précédent \ref{conception}. Grâce à cela, 
  l'interface utilisateur, la gestion du réseau et le moteur \emph{logoot}
  sont totalement découplés les uns des autres. L'objectif de départ étant
  de fournir deux interfaces : une en client leger (web avec \emph{Javascript})
  et une en client lourd (avec \emph{Swing}\footnote{Swing est une bibliothèque
  graphique pour \emph{Java}}). 
  
  \subsection{État d'avancement}
	L'implentation est complétement fonctionnel mais nous nous sommes heurté
	a de nombreux problèmes.
	
	\label{sec:textarea}
	Pour implémenter l'éditeur, on utilise un \emph{TextArea}. C'est le 
	composant de base de \emph{html} pour ce genre de tache. Cet élément 
	est trés contraignant. En effet la gestion du curseur est complexe
	pour des raisons de sécurité ( afin d'éviter une introsion trop forte).
	Afin de changer le texte quand des changement sont effectué à 
	l'exterieur, nous somme obligé de redéfinir tout le texte après un 
	passage dans un algorithme de \emph{diff}\footnote{permet la comparaison 
	de texte en trouvant les différences}/\emph{patch}\footnote{integration
	d'un resultat de diff pour introduire les modifications}. Ce procédé
	est lent et a tendance à nous faire perdre certains caractères lorsque
	l'on tape rapidement.
	
	Le protocol clasique de communication utilisé sur le web n'est pas 
	adapté à notre projet. En général, ce sont les pages web qui requêtent
	le serveur et celui-ci réponds. Aucun problème pour envoyer nos modifications 
	au serveur via un appel RPC fournis par \emph{GWT}. Le problème se présente
	pour permettre la réception des modifications. Plusieurs solutions 
	techniques sont disponible :
	\begin{itemize}
		\item \emph{Server Push}
		\item \emph{Web Socket}
		\item \emph{Server Sent Event}
	\end{itemize}
