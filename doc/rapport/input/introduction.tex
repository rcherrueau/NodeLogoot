\chapter*{Introduction}
\addcontentsline{toc}{chapter}{Introduction}

De nos jours, le travail collaboratif a une place importante. En effet, le monde
de l'entreprise et de la recherche oblige de plus en plus le travail à distance
entre plusieurs personnes en même temps. Une aide pour ce travail est bien
entendu les vidéos-conférences dans lesquelles plusieurs personnes distantes
peuvent communiquer en direct et sans problème. Pouvoir éditer des documents en
même temps et par tous est un atout majeur dans le travail collaboratif. C'est
ce que nous propose \emph{Google Docs}.

\emph{Google Docs} permet à des utilisateurs de pouvoir travailler sur un même
document, en même temps et en voyant les modifications effectuées par les autres
utilisateurs directement sur votre écran. Cependant, cette solution possède
quelques limites, notamment du fait que le document partagé est centralisé sur
un serveur distant et est donc retransmit à tous les autres utilisateurs.

L'objectif de ce projet est d'apporter une solution à cette lacune en
distribuant le document partagé via un réseau pair-à-pair, de ce fait, chaque
utilisateur possède l'ensemble du document et seulement les modifications
apportées à ce dernier seront transmises aux autres utilisateurs qui se mettront
à jour. Ceci permet de limiter les risques de perte du document (chacun possède
le document et non pas un serveur centralisé) et les modifications apportées à
ce dernier seront transmises beaucoup plus rapidement. De plus, le gain
significatif au niveau des coûts (pas besoin de serveur, rapidité, ...) est un
argument considérable d'un point de vue économique.

L'approche de \emph{Logoot} ne limite pas le nombre d'utilisateur ni le nombre
d'édition du document partagé tout en garantissant la causalité, la consistance
et la préservation de l'intention.

Ce document présente l'algorithme \emph{Logoot}, puis le travail effectué autour
de celui-ci (développement d'outils d'édition collaborative).

