\section{Implémentation en \emph{Dart}}
  \subsection{Qu'est-ce que \emph{Dart}}
    \emph{Dart} est une nouvelle plate-forme développée par \emph{Google}. Cette
    platte-forme est conçue pour pouvoir faire des applications Web structurées.
    Il y a deux manières d'exécuter du code \emph{Dart} :
    \begin{enumerate}
      \item Compilation en \emph{Javascript} d'une application.
      \item Exécution d'une application au sein d'une \emph{VM}\footnote{Virtual
            Machine} \emph{Dart}. Cette solution est fonctionnelle dans un
            navigateur Web, mais elle a aussi la particularité d'être
            fonctionnelle au sein d'un serveur.
    \end{enumerate}
    \emph{Dart} est donc un langage permettant de développer une application Web
    structuré dans son intégralité (à la fois le client et le serveur).

    Cette technologie est prometteuse mais encore trop jeune et donc pas assez
    mature.

  \subsection{Objectif de l'implémentation en \emph{Dart}}
    \emph{Dart} étant une technologie prometteuse mais encore pas suffisamment
    mature, l'objectif de cette alternative \emph{Dart} est de déterminer s'il
    est possible d'implémenter un éditeur collaboratif utilisant \emph{Logoot}
    en \emph{Dart}.

  \subsection{État d'avancement}
    L'implémentation de \emph{Logoot} en \emph{Dart} est fonctionnelle mais il
    manque encore les fonctionnalités permettant la communication avec
    l'extérieur. Les objets \emph{EventSource} et \emph{WebSocket} sont
    notamment prévus mais pas encore disponibles.

