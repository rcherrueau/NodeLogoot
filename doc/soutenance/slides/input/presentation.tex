\section{Présentation de Logoot}
  \begin{frame}
    \frametitle{Présentation de Logoot}
    \begin{itemize}
      \item Algorithme permettant la génération d'identifiants pour le partage des modifications entre les personnes travaillant sur le document.
      \item Identifiants permettant le positionnement des modifications dans le document.
    \end{itemize}
  \end{frame}

\begin{frame}
    \frametitle{Principe de Logoot}
    \begin{itemize}
      \item Table des identifiants.
      \begin{itemize}
      	\item Position de chaque caractère du document.
      	\item Identifié par un \emph{identifiant de caractère} unique.
      	\item Liste des nouveaux identifiants de caractère mis-à-jour par tous les utilisateurs du document.
      \end{itemize}
    \end{itemize}
  \end{frame}

\begin{frame}  
  \frametitle{Identifiant de position}
	\begin{center}
		\textbf{<i, s, h>}
	\end{center}
		\begin{itemize}
			\item \emph{i} : un chiffre dans la base \emph{BASE}. La valeur
			maximum que peut prendre \emph{i} \emph{MAX} sera entre $0$ et
			\emph{BASE}$ - 1$.
			\item \emph{s} : un identifiant unique de la réplique.
			\item \emph{h} : la valeur de l'horloge logique de la réplique.
		\end{itemize}
  \end{frame}

\begin{frame}  
  \frametitle{Identifiant de caractère}
  	Liste ordonnée non-mutable d'identifiants de positions\\
  	Exemples:
	\begin{center}
		\textbf{<42, 1, 1><1337, 1, 72>}\\
		\textbf{<2, 4, 4>}
	\end{center}
  \end{frame}
  
  \begin{frame}  
  \frametitle{Table des identifiants}
  	Stocke les identifiants de	caractères.
  \end{frame}


\begin{frame}  
  \frametitle{Exemple d'insertion}
  	\begin{center}
		\textbf{ab}
	\end{center}
  	\begin{table}
			\center
			\begin{tabular}{|l|l|l|}			
			\hline
				Table des identifiants & Identifiant de caractère & Caractère\\
			\hline
				0 & \verb+<0, NA, NA>+ & NA\\
				1 & \verb+<41, 3, 1>+ & a\\
				2 & \verb+<44, 2, 16>+ & b\\
				MAX & \verb+<MAX, NA, NA>+ & NA\\
			\hline
			\end{tabular}
			\caption{Exemple de table des identifiants - Origine}
		\end{table}
	\end{frame}
	
\begin{frame}  
  \frametitle{Exemple d'insertion}
  	Insertion de la lettre \textbf{e} entre \textbf{a} et \textbf{b}.\\
  	Entre les identifiants:
  	\textbf{<41, 3, 1>} et \textbf{<44, 2, 16>}.
  	\begin{itemize}
		\item On regarde si on a la <<place>> pour placer les nouveaux
			identifiants\\ \textbf{intervalle = 44 - 41 - 1 = 2}. 
			Il y a donc la place pour générer deux identifiants.
		\item Un identifiant est généré dans l'intervalle \textbf{]41, 44[}.
		\item Insertion de l'identifiant généré \textbf{<42, 1, 1>} dans la table des identifiants.
	\end{itemize}
\end{frame}

\begin{frame}	
	\frametitle{Exemple d'insertion}
  	\begin{center}
		\textbf{aeb}
	\end{center}
  	\begin{table}
			\center
			\begin{tabular}{|l|l|l|}			
			\hline
				Table des identifiants & Identifiant de caractère & Caractère\\
			\hline
				0 & \verb+<0, NA, NA>+ & NA\\
				1 & \verb+<41, 3, 1>+ & a\\
				2 & \verb+<42, 1, 1>+ & e\\
				3 & \verb+<44, 2, 16>+ & b\\
				MAX & \verb+<MAX, NA, NA>+ & NA\\
			\hline
			\end{tabular}
			\caption{Exemple de table des identifiants - Insertion de 'e'}
		\end{table}
	\end{frame}