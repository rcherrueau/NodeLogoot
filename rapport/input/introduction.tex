\chapter*{Introduction}
\addcontentsline{toc}{chapter}{Introduction}

De nos jours, le travail collaboratif possède une place importante. En effet, le
monde de l'entreprise et de la recherche oblige de plus en plus le travail à
distance entre plusieurs personnes en même temps. Une aide pour ce travail est
bien entendu les vidéos-conférences dans lesquelles plusieurs personnes
distantes peuvent communiquer en direct et sans problème. Pouvoir éditer des
documents en même temps et par tous est un atout majeur dans le travail
collaboratif. C'est ce que nous propose \emph{Google Docs}.

En effet, \emph{Google Docs} permet à des utilisateurs de pouvoir travailler sur
un même document, en même temps et en voyant les modifications effectuées par
les autres utilisateurs directement sur votre écran. Cependant, cette solution
possède quelques limites, notamment du fait que le document partagé est
centralisé sur un serveur distant et est donc retransmit à tous les autres
utilisateurs.

La solution apportée à cette lacune est de distribuer le document partagé en
utilisant un réseau peer-to-peer\footnote{Réseau pair-à-pair}, de ce
fait, chaque utilisateur possède l'ensemble du document et seulement les
modifications apportées à ce dernier seront transmis aux autres utilisateurs qui
se mettront à jour. Ceci permet de limiter les risques de perte du document
(chacun possède le document et non pas un serveur centralisé) et les
modifications apportées à ce dernier seront transmises beaucoup plus rapidement.

L'approche de \emph{Logoot} ne limite pas le nombre d'utilisateur ni le nombre
d'édition du document partagé tout en garantissant la causalité, la consistance
et la préservation de l'intention.

Ce document présente \emph{Logoot}, sa description, son utilisation ainsi que
l'outil développé afin de faire du travail collaboratif en utilisant l'approche
\emph{Logoot}.